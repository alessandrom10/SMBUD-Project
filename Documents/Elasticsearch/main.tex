% A LaTeX template for MSc Thesis submissions to 
% Politecnico di Milano (PoliMi) - School of Industrial and Information Engineering
%
% S. Bonetti, A. Gruttadauria, G. Mescolini, A. Zingaro
% e-mail: template-tesi-ingind@polimi.it
%
% Last Revision: October 2021
%
% Copyright 2021 Politecnico di Milano, Italy. NC-BY

\documentclass{Configuration_Files/PoliMi3i_thesis}

%------------------------------------------------------------------------------
%	REQUIRED PACKAGES AND  CONFIGURATIONS
%------------------------------------------------------------------------------

% CONFIGURATIONS
\usepackage{parskip} % For paragraph layout
\usepackage{setspace} % For using single or double spacing
\usepackage{emptypage} % To insert empty pages
\usepackage{multicol} % To write in multiple columns (executive summary)
\setlength\columnsep{15pt} % Column separation in executive summary
\setlength\parindent{0pt} % Indentation
\raggedbottom  

% PACKAGES FOR TITLES
\usepackage{titlesec}
% \titlespacing{\section}{left spacing}{before spacing}{after spacing}
\titlespacing{\section}{0pt}{3.3ex}{2ex}
\titlespacing{\subsection}{0pt}{3.3ex}{1.65ex}
\titlespacing{\subsubsection}{0pt}{3.3ex}{1ex}
\usepackage{color}

% PACKAGES FOR LANGUAGE AND FONT
\usepackage[english]{babel} % The document is in English  
\usepackage[utf8]{inputenc} % UTF8 encoding
\usepackage[T1]{fontenc} % Font encoding
\usepackage[11pt]{moresize} % Big fonts

% PACKAGES FOR IMAGES
\usepackage{graphicx}
\usepackage{transparent} % Enables transparent images
\usepackage{eso-pic} % For the background picture on the title page
\usepackage{subfig} % Numbered and caption subfigures using \subfloat.
\usepackage{tikz} % A package for high-quality hand-made figures.
\usetikzlibrary{}
\graphicspath{{./Images/}} % Directory of the images
\usepackage{caption} % Coloured captions
\usepackage{xcolor} % Coloured captions
\usepackage{amsthm,thmtools,xcolor} % Coloured "Theorem"
\usepackage{float}

% STANDARD MATH PACKAGES
\usepackage{amsmath}
\usepackage{amsthm}
\usepackage{amssymb}
\usepackage{amsfonts}
\usepackage{bm}
\usepackage[overload]{empheq} % For braced-style systems of equations.
\usepackage{fix-cm} % To override original LaTeX restrictions on sizes

% PACKAGES FOR TABLES
\usepackage{tabularx}
\usepackage{longtable} % Tables that can span several pages
\usepackage{colortbl}

% PACKAGES FOR ALGORITHMS (PSEUDO-CODE)
\usepackage{algorithm}
\usepackage{algorithmic}
\usepackage{algpseudocode}

% PACKAGES FOR REFERENCES & BIBLIOGRAPHY
\usepackage[colorlinks=true,linkcolor=black,anchorcolor=black,citecolor=black,filecolor=black,menucolor=black,runcolor=black,urlcolor=black]{hyperref} % Adds clickable links at references
\usepackage{cleveref}
\usepackage[square, numbers, sort&compress]{natbib} % Square brackets, citing references with numbers, citations sorted by appearance in the text and compressed
\bibliographystyle{abbrvnat} % You may use a different style adapted to your field

% OTHER PACKAGES
\usepackage{pdfpages} % To include a pdf file
\usepackage{afterpage}
\usepackage{lipsum} % DUMMY PACKAGE
\usepackage{fancyhdr} % For the headers
\fancyhf{}

% Input of configuration file. Do not change config.tex file unless you really know what you are doing. 
% Define blue color typical of polimi
\definecolor{bluepoli}{cmyk}{0.4,0.1,0,0.4}

% Custom theorem environments
\declaretheoremstyle[
  headfont=\color{bluepoli}\normalfont\bfseries,
  bodyfont=\color{black}\normalfont\itshape,
]{colored}

% Set-up caption colors
\captionsetup[figure]{labelfont={color=bluepoli}} % Set colour of the captions
\captionsetup[table]{labelfont={color=bluepoli}} % Set colour of the captions
\captionsetup[algorithm]{labelfont={color=bluepoli}} % Set colour of the captions

\theoremstyle{colored}
\newtheorem{theorem}{Theorem}[chapter]
\newtheorem{proposition}{Proposition}[chapter]

% Enhances the features of the standard "table" and "tabular" environments.
\newcommand\T{\rule{0pt}{2.6ex}}
\newcommand\B{\rule[-1.2ex]{0pt}{0pt}}

% Pseudo-code algorithm descriptions.
\newcounter{algsubstate}
\renewcommand{\thealgsubstate}{\alph{algsubstate}}
\newenvironment{algsubstates}
  {\setcounter{algsubstate}{0}%
   \renewcommand{\STATE}{%
     \stepcounter{algsubstate}%
     \Statex {\small\thealgsubstate:}\space}}
  {}

% New font size
\newcommand\numfontsize{\@setfontsize\Huge{200}{60}}

% Title format: chapter
\titleformat{\chapter}[hang]{
\fontsize{50}{20}\selectfont\bfseries\filright}{\textcolor{bluepoli} \thechapter\hsp\hspace{2mm}\textcolor{bluepoli}{|   }\hsp}{0pt}{\huge\bfseries \textcolor{bluepoli}
}

% Title format: section
\titleformat{\section}
{\color{bluepoli}\normalfont\Large\bfseries}
{\color{bluepoli}\thesection.}{1em}{}

% Title format: subsection
\titleformat{\subsection}
{\color{bluepoli}\normalfont\large\bfseries}
{\color{bluepoli}\thesubsection.}{1em}{}

% Title format: subsubsection
\titleformat{\subsubsection}
{\color{bluepoli}\normalfont\large\bfseries}
{\color{bluepoli}\thesubsubsection.}{1em}{}

% Shortening for setting no horizontal-spacing
\newcommand{\hsp}{\hspace{0pt}}

\makeatletter
% Renewcommand: cleardoublepage including the background pic
\renewcommand*\cleardoublepage{%
  \clearpage\if@twoside\ifodd\c@page\else
  \null
  \AddToShipoutPicture*{\BackgroundPic}
  \thispagestyle{empty}%
  \newpage
  \if@twocolumn\hbox{}\newpage\fi\fi\fi}
\makeatother

%For correctly numbering algorithms
\numberwithin{algorithm}{chapter}

%----------------------------------------------------------------------------
%	NEW COMMANDS DEFINED
%----------------------------------------------------------------------------

% EXAMPLES OF NEW COMMANDS
\newcommand{\bea}{\begin{eqnarray}} % Shortcut for equation arrays
\newcommand{\eea}{\end{eqnarray}}
\newcommand{\e}[1]{\times 10^{#1}}  % Powers of 10 notation

%----------------------------------------------------------------------------
%	ADD YOUR PACKAGES (be careful of package interaction)
%----------------------------------------------------------------------------
\usepackage{listings}
\usepackage{xcolor}
\usepackage{geometry}

%----------------------------------------------------------------------------
%	ADD YOUR DEFINITIONS AND COMMANDS (be careful of existing commands)
%----------------------------------------------------------------------------

%----------------------------------------------------------------------------
%	BEGIN OF YOUR DOCUMENT
%----------------------------------------------------------------------------

\begin{document}

\fancypagestyle{plain}{%
\fancyhf{} % Clear all header and footer fields
\fancyhead[RO,RE]{\thepage} %RO=right odd, RE=right even
\renewcommand{\headrulewidth}{0pt}
\renewcommand{\footrulewidth}{0pt}}

%----------------------------------------------------------------------------
%	TITLE PAGE
%----------------------------------------------------------------------------

\pagestyle{empty} % No page numbers
\frontmatter % Use roman page numbering style (i, ii, iii, iv...) for the preamble pages

\puttitle{
	title=Systems and Methods for Big and Unstructured Data Project,
	name1=Matteo Balice - 10978268, % Author Name and Surname
	name2=Antonio Giuseppe Doronzo - 11016435, 
	name3=Alessandro Masini - 10940986, 
	name4=,
	name5=,
	academicyear=2023-2024
} % These info will be put into your Title page 

%----------------------------------------------------------------------------
%	PREAMBLE PAGES: ABSTRACT (inglese e italiano), EXECUTIVE SUMMARY
%----------------------------------------------------------------------------
\startpreamble
\setcounter{page}{1} % Set page counter to 1

%----------------------------------------------------------------------------
%	LIST OF CONTENTS/FIGURES/TABLES/SYMBOLS
%----------------------------------------------------------------------------

% TABLE OF CONTENTS
\thispagestyle{empty}
\tableofcontents % Table of contents 
\thispagestyle{empty}
\cleardoublepage

%-------------------------------------------------------------------------
%	THESIS MAIN TEXT
%-------------------------------------------------------------------------
% In the main text of your thesis you can write the chapters in two different ways:
%
%(1) As presented in this template you can write:
%    \chapter{Title of the chapter}
%    *body of the chapter*
%
%(2) You can write your chapter in a separated .tex file and then include it in the main file with the following command:
%    \chapter{Title of the chapter}
%    \input{chapter_file.tex}
%
% Especially for long thesis, we recommend you the second option.

\addtocontents{toc}{\vspace{2em}} % Add a gap in the Contents, for aesthetics
\mainmatter % Begin numeric (1,2,3...) page numbering

\chapter{Elasticsearch}
\label{ch:mongodb}%
% The \label{...}% enables to remove the small indentation that is generated, always leave the % symbol.

\section{Introduction}
Using this dataset, the main objective was to understand the dynamics of spreading fake news by analyzing the available data in detail. To address this challenge and obtain meaningful insights, Elasticsearch, a powerful distributed search engine, was adopted for its ability to manage and analyze large volumes of data efficiently.

The use of Elasticsearch was motivated by several key factors. First, its flexibility allowed me to efficiently model and index my dataset, enabling me to perform complex queries on large datasets. Elasticsearch's JSON-based structure was particularly well suited to represent the information contained in the fake news dataset in a clear and structured manner.

In addition, Elasticsearch's powerful aggregation functionality was leveraged to extract meaningful insights from my analysis. The ability to dynamically group, filter, and aggregate data facilitated the discovery of patterns and trends in fake news, providing a detailed picture of their distinguishing characteristics.

\section{Dataset}
This dataset provides detailed information on fake news, collecting each news item as a single document easily identifiable by its name. You can quickly locate each news item using the dataset's search function. \\
The data schema is:
\begin{table}[h!]
	\begin{center}
		\begin{tabular}{|m{6em}|m{4em}|m{20em}|}
		\hline
		\textbf{Attribute} & \textbf{Type} & \textbf{Description}\\
		\hline
			title & String & The title of the news.\\
		\hline
			text & String & The text of the news.\\
            \hline
			subject & String & The category of the news.\\
		\hline
                date & String & The date of the news.\\
            \hline
                is\_fake & Boolean & A boolean value that states if a news is fake or not.\\
            \hline
		\end{tabular}
	\end{center}
\end{table}
The source of the dataset is:\\
\url{https://www.kaggle.com/datasets/bhavikjikadara/fake-news-detection}
\newpage
\section*{Queries}
\subsection{List of subjects}
This query returns an aggregate analysis of the topics present in the documents. Aggregation is done on the "subject" field, providing a list of distinct topics present in the dataset and the number of times each topic appears.\\

\begin{algorithm}[ht]
\caption{List of subjects}
\begin{lstlisting} [numbers = left]
GET /fake-news/_search
{ 
  "size": 0,
  "aggs": {
    "categories": {
      "terms": {
        "field": "subject"
      }
    }
  }
}
\end{lstlisting}
\end{algorithm}
\newpage

Output:
\begin{algorithm}[h!]
\caption{Output List of subjects}
\begin{lstlisting} [numbers = left]
{
  "took": 35,
  "timed_out": false,
  "_shards": {
    "total": 1,
    "successful": 1,
    "skipped": 0,
    "failed": 0
  },
  "hits": {
    "total": {
      "value": 10000,
      "relation": "gte"
    },
    "max_score": null,
    "hits": []
  },
  
  "aggregations": {
    "categories": {
      "doc_count_error_upper_bound": 0,
      "sum_other_doc_count": 59,
      "buckets": [
        {
          "key": "politicsNews",
          "doc_count": 8979
        },
        {
          "key": "worldnews",
          "doc_count": 8100
        },
        {
          "key": "News",
          "doc_count": 7261
        },
        {
          "key": "politics",
          "doc_count": 5147
        },

\end{lstlisting}
\end{algorithm}
\newpage
\begin{algorithm}[h!]
\caption{Output List of subjects}
\begin{lstlisting} [numbers = left]
        {
          "key": "left-news",
          "doc_count": 3445
        },
        {
          "key": "Government News",
          "doc_count": 1204
        },
        {
          "key": "US_News",
          "doc_count": 645
        },
        {
          "key": "Middle-east",
          "doc_count": 622
        },
        {
          "key": ",politics",
          "doc_count": 333
        },
        {
          "key": ",left-news",
          "doc_count": 120
        }
      ]
    }
  }
}
\end{lstlisting}
\end{algorithm}
\newpage
\subsection{Vaccine fake/true news}
This query will return the aggregate count of documents in the "fake-news" dataset that contain the word "vaccine" in the "text" field, splitting the count based on whether the news is marked as fake or not.\\

\begin{algorithm}[ht]
\caption{Vaccine fake/true news}
\begin{lstlisting} [numbers = left]
GET /fake-news/_search
{
  "size": 0,
  "query": {
    "bool": {
      "must": [
        {
          "match": {
            "text": "vaccine"
          }
        }
      ]
    }
  },
  "aggs": {
    "fake_news_count": {
      "terms": {
        "field": "is_fake"
      }
    }
  }
}
\end{lstlisting}
\end{algorithm}
\newpage

Output:
\begin{algorithm}[h!]
\caption{Vaccine fake/true news}
\begin{lstlisting} [numbers = left]
{
  "took": 3,
  "timed_out": false,
  "_shards": {
    "total": 1,
    "successful": 1,
    "skipped": 0,
    "failed": 0
  },
  "hits": {
    "total": {
      "value": 42,
      "relation": "eq"
    },
    "max_score": null,
    "hits": []
  },
  "aggregations": {
    "fake_news_count": {
      "doc_count_error_upper_bound": 0,
      "sum_other_doc_count": 0,
      "buckets": [
        {
          "key": 0,
          "doc_count": 29
        },
        {
          "key": 1,
          "doc_count": 13
        }
      ]
    }
  }
}

\end{lstlisting}
\end{algorithm}
\newpage
\newpage
\subsection{2017 fake/true news}
this query will return the aggregate count of documents in the "fake-news" dataset that have a date containing the year "2017," splitting the count according to whether the news is marked as fake or not.\\

\begin{algorithm}[ht]
\caption{2017 fake/true news}
\begin{lstlisting} [numbers = left]
GET /fake-news/_search
{
  "size": 0,
  "query": {
    "bool": {
      "must": [
        {
          "wildcard": {
            "date": "*2017*"
          }
        }
      ]
    }
  },
  "aggs": {
    "fake_news_count": {
      "terms": {
        "field": "is_fake"
      }
    }
  }
}
\end{lstlisting}
\end{algorithm}
\newpage

Output:
\begin{algorithm}[h!]
\caption{2017 fake/true news}
\begin{lstlisting} [numbers = left]
{
  "took": 25,
  "timed_out": false,
  "_shards": {
    "total": 1,
    "successful": 1,
    "skipped": 0,
    "failed": 0
  },
  "hits": {
    "total": {
      "value": 10000,
      "relation": "gte"
    },
    "max_score": null,
    "hits": []
  },
  "aggregations": {
    "fake_news_count": {
      "doc_count_error_upper_bound": 0,
      "sum_other_doc_count": 0,
      "buckets": [
        {
          "key": 0,
          "doc_count": 13359
        },
        {
          "key": 1,
          "doc_count": 7358
        }
      ]
    }
  }
}

\end{lstlisting}
\end{algorithm}
\newpage
\subsection{Trump 2016 fake/true news}
this query will return the aggregate count of the documents in the "fake-news" dataset that contain the word "Trump" in the "text" field and have a date containing the year "2016," the year of his election as President of the United States of America, splitting the count based on whether the news is marked as fake or not.\\

\begin{algorithm}[ht]
\caption{Trump 2016 fake/true news}
\begin{lstlisting} [numbers = left]
GET /fake-news/_search
{ 
  "size": 0,
  "query": {
    "bool": {
      "must": [
        { "match": { "text": "Trump" }},
        { "wildcard": { "date": "*2016*" }}
      ]
    }
  },
  "aggs": {
    "trump_news": {
      "terms": {
        "field": "is_fake"
      }
    }
  }
}
\end{lstlisting}
\end{algorithm}
\newpage

Output:
\begin{algorithm}[h!]
\caption{Trump 2016 fake/true news}
\begin{lstlisting} [numbers = left]
{
  "took": 11,
  "timed_out": false,
  "_shards": {
    "total": 1,
    "successful": 1,
    "skipped": 0,
    "failed": 0
  },
  "hits": {
    "total": {
      "value": 6647,
      "relation": "eq"
    },
    "max_score": null,
    "hits": []
  },
  "aggregations": {
    "trump_news": {
      "doc_count_error_upper_bound": 0,
      "sum_other_doc_count": 0,
      "buckets": [
        {
          "key": 1,
          "doc_count": 4554
        },
        {
          "key": 0,
          "doc_count": 2093
        }
      ]
    }
  }
}

\end{lstlisting}
\end{algorithm}
\newpage
\subsection{Subject fake/true news}
This query will return an aggregate analysis of the topics in the "fake-news" documents, showing the number of news items for each topic and further breaking this count down by whether the news item is marked as fake or not..\\

\begin{algorithm}[ht]
\caption{Subject fake/true news}
\begin{lstlisting} [numbers = left]
GET /fake-news/_search
{ 
  "size": 0,
  "aggs": {
    "categories": {
      "terms": {
        "field": "subject"
      },
      "aggs": {
        "is_fake": {
          "terms": {
            "field": "is_fake"
          }
        }
      }
    }
  }
}
\end{lstlisting}
\end{algorithm}
\newpage

Output:
\begin{algorithm}[h!]
\caption{Subject fake/true news}
\begin{lstlisting} [numbers = left]
{
  "took": 10,
  "timed_out": false,
  "_shards": {
    "total": 1,
    "successful": 1,
    "skipped": 0,
    "failed": 0
  },
  "hits": {
    "total": {
      "value": 10000,
      "relation": "gte"
    },
    "max_score": null,
    "hits": []
  },
  "aggregations": {
    "categories": {
      "doc_count_error_upper_bound": 0,
      "sum_other_doc_count": 59,
      "buckets": [
        {
          "key": "politicsNews",
          "doc_count": 8979,
          "is_fake": {
            "doc_count_error_upper_bound": 0,
            "sum_other_doc_count": 0,
            "buckets": [
              {
                "key": 0,
                "doc_count": 8979
              }
            ]
          }
        },

\end{lstlisting}
\end{algorithm}
\newpage
\begin{algorithm}[h!]
\caption{Subject fake/true news}
\begin{lstlisting} [numbers = left]
        {
          "key": "worldnews",
          "doc_count": 8100,
          "is_fake": {
            "doc_count_error_upper_bound": 0,
            "sum_other_doc_count": 0,
            "buckets": [
              {
                "key": 0,
                "doc_count": 8100
              }
            ]
          }
        },
        {
          "key": "News",
          "doc_count": 7261,
          "is_fake": {
            "doc_count_error_upper_bound": 0,
            "sum_other_doc_count": 0,
            "buckets": [
              {
                "key": 1,
                "doc_count": 7261
              }
            ]
          }
        },
        {
          "key": "politics",
          "doc_count": 5147,
          "is_fake": {
            "doc_count_error_upper_bound": 0,
            "sum_other_doc_count": 0,
            "buckets": [
              {
                "key": 1,
                "doc_count": 5147
              }
            ]
          }
        },
\end{lstlisting}
\end{algorithm}
\newpage
\begin{algorithm}[h!]
\caption{Subject fake/true news}
\begin{lstlisting} [numbers = left]
        {
          "key": "left-news",
          "doc_count": 3445,
          "is_fake": {
            "doc_count_error_upper_bound": 0,
            "sum_other_doc_count": 0,
            "buckets": [
              {
                "key": 1,
                "doc_count": 3445
              }
            ]
          }
        },
        {
          "key": "Government News",
          "doc_count": 1204,
          "is_fake": {
            "doc_count_error_upper_bound": 0,
            "sum_other_doc_count": 0,
            "buckets": [
              {
                "key": 1,
                "doc_count": 1204
              }
            ]
          }
        },
        {
          "key": "US_News",
          "doc_count": 645,
          "is_fake": {
            "doc_count_error_upper_bound": 0,
            "sum_other_doc_count": 0,
            "buckets": [
              {
                "key": 1,
                "doc_count": 645
              }
            ]
          }
        },
\end{lstlisting}
\end{algorithm}
\newpage
\begin{algorithm}[h!]
\caption{Subject fake/true news}
\begin{lstlisting} [numbers = left]
        {
          "key": "Middle-east",
          "doc_count": 622,
          "is_fake": {
            "doc_count_error_upper_bound": 0,
            "sum_other_doc_count": 0,
            "buckets": [
              {
                "key": 1,
                "doc_count": 622
              }
            ]
          }
        },
        {
          "key": ",politics",
          "doc_count": 333,
          "is_fake": {
            "doc_count_error_upper_bound": 0,
            "sum_other_doc_count": 0,
            "buckets": [
              {
                "key": 1,
                "doc_count": 333
              }
            ]
          }
        },
        {
          "key": ",left-news",
          "doc_count": 120,
          "is_fake": {
            "doc_count_error_upper_bound": 0,
            "sum_other_doc_count": 0,
            "buckets": [
              {
                "key": 1,
                "doc_count": 120
              }
            ]
          }
        }
\end{lstlisting}
\end{algorithm}
\subsection{True news}
This query will return documents from the "fake-news" dataset in which the "is\_fake" field is set to 0, i.e., documents that could be considered as not fake-news.\\

\begin{algorithm}[ht]
\caption{True news}
\begin{lstlisting} [numbers = left]
GET /fake-news/_search
{  
  "size": 3,
  "query": {
    "term": {
      "is_fake": 0
    }
  }
}
\end{lstlisting}
\end{algorithm}
\newpage

Output:
\begin{algorithm}[h!]
\caption{True news}
\begin{lstlisting} [numbers = left]
{
  "took": 31,
  "timed_out": false,
  "_shards": {
    "total": 1,
    "successful": 1,
    "skipped": 0,
    "failed": 0
  },
  "hits": {
    "total": {
      "value": 10000,
      "relation": "gte"
    },
    "max_score": 1,
    "hits": [
      {
        "_index": "fake-news",
        "_id": "SYHnvIwBSKJFig-fn3dZ",
        "_score": 1,
        "_source": {
          "date": "December 10, 2017 ",
          "text": "BERLIN (Reuters) - Germany s intelligence service...",
          "title": "German intelligence unmasks alleged covert Chinese social media profiles",
          "is_fake": 0,
          "subject": "worldnews"
        }
      },
\end{lstlisting}
\end{algorithm}
\newpage
\begin{algorithm}[h!]
\caption{True news}
\begin{lstlisting} [numbers = left]
      {
        "_index": "fake-news",
        "_id": "SoHnvIwBSKJFig-fn3dZ",
        "_score": 1,
        "_source": {
          "date": "September 2, 2017 ",
          "text": "HOUSTON (Reuters) - Union Pacific Corp...",
          "title": "Union Pacific says Arkema chemical plant fire hindering line repairs",
          "is_fake": 0,
          "subject": "politicsNews"
        }
      },
      {
        "_index": "fake-news",
        "_id": "S4HnvIwBSKJFig-fn3dZ",
        "_score": 1,
        "_source": {
          "date": "June 30, 2016 ",
          "text": "OTTAWA (Reuters) - U.S. President Barack Obama...",
          "title": "Obama: Trump's rhetoric is xenophobic, not populist",
          "is_fake": 0,
          "subject": "politicsNews"
        }
      },

\end{lstlisting}
\end{algorithm}
\newpage
\subsection{2016/2017/2018 fake/true news}
This query will return an aggregate analysis of "fake-news" documents, showing the number of news items per year (extracted from the date) and further breaking this count down by whether the news item is marked as fake or not. The size is limited to 3, so data will only be returned for the first three years.\\

\begin{algorithm}[ht]
\caption{2016/2017/2018 fake/true news}
\begin{lstlisting} [numbers = left]
GET /fake-news/_search
{
  "size": 0,
  "aggs": {
    "categories": {
      "terms": {
        "script": {
          "source": "doc['date'].value.trim().substring(doc['date']
             .value.trim().length() - 4)",
          "lang": "painless"
        },
        "size": 3
      },
      "aggs": {
        "is_fake": {
          "terms": {
            "field": "is_fake"
          }
        }
      }
    }
  }
}
\end{lstlisting}
\end{algorithm}
\newpage

Output:
\begin{algorithm}[h!]
\caption{2016/2017/2018 fake/true news}
\begin{lstlisting} [numbers = left]
{
  "took": 71,
  "timed_out": false,
  "_shards": {
    "total": 1,
    "successful": 1,
    "skipped": 0,
    "failed": 0
  },
  "hits": {
    "total": {
      "value": 10000,
      "relation": "gte"
    },
    "max_score": null,
    "hits": []
  },
  "aggregations": {
    "categories": {
      "doc_count_error_upper_bound": 0,
      "sum_other_doc_count": 31,
      "buckets": [
        {
          "key": "2017",
          "doc_count": 20717,
          "is_fake": {
            "doc_count_error_upper_bound": 0,
            "sum_other_doc_count": 0,
            "buckets": [
              {
                "key": 0,
                "doc_count": 13359
              },
              {
                "key": 1,
                "doc_count": 7358
              }
            ]
          }
        },

\end{lstlisting}
\end{algorithm}
\newpage
\begin{algorithm}[h!]
\caption{2016/2017/2018 fake/true news}
\begin{lstlisting} [numbers = left]
        {
          "key": "2016",
          "doc_count": 13162,
          "is_fake": {
            "doc_count_error_upper_bound": 0,
            "sum_other_doc_count": 0,
            "buckets": [
              {
                "key": 1,
                "doc_count": 9442
              },
              {
                "key": 0,
                "doc_count": 3720
              }
            ]
          }
        },
        {
          "key": "2015",
          "doc_count": 2005,
          "is_fake": {
            "doc_count_error_upper_bound": 0,
            "sum_other_doc_count": 0,
            "buckets": [
              {
                "key": 1,
                "doc_count": 2005
              }
            ]
          }
        }
      ]
    }
  }
}

\end{lstlisting}
\end{algorithm}
\newpage
\subsection{Pope news}
This query will return documents from the "fake-news" dataset in which the "text" field contains the word "Pope".\\

\begin{algorithm}[ht]
\caption{Pope news}
\begin{lstlisting} [numbers = left]
GET /fake-news/_search
{ 
  "size": 3,
  "query": {
    "bool": {
      "must": [
        {
          "match": {
            "text": "Pope"
          }
        }
      ]
    }
  }
}
\end{lstlisting}
\end{algorithm}
\newpage

Output:
\begin{algorithm}[h!]
\caption{Pope news}
\begin{lstlisting} [numbers = left]
{
  "took": 10,
  "timed_out": false,
  "_shards": {
    "total": 1,
    "successful": 1,
    "skipped": 0,
    "failed": 0
  },
  "hits": {
    "total": {
      "value": 259,
      "relation": "eq"
    },
    "max_score": 9.928849,
    "hits": [
      {
        "_index": "fake-news",
        "_id": "QoHnvIwBSKJFig-fpoJb",
        "_score": 9.928849,
        "_source": {
          "date": "Jul 9, 2017",
          "text": "The last time a pope got involved in politics was when...",
          "title": "POPE FRANCIS Worries USA Has Distorted Vision Of The World",
          "is_fake": 1,
          "subject": "left-news"
        }
      },

\end{lstlisting}
\end{algorithm}
\newpage
\begin{algorithm}[h!]
\caption{Pope news}
\begin{lstlisting} [numbers = left]
      {
        "_index": "fake-news",
        "_id": "VYHnvIwBSKJFig-f3uKJ",
        "_score": 9.875764,
        "_source": {
          "date": "February 19, 2016",
          "text": "Donald Trump has added Pope Francis to his...",
          "title": " Colbert Lays A Catholic Smack Down On Trump Over Pope Francis Fight (VIDEO)",
          "is_fake": 1,
          "subject": "News"
        }
      },
      {
        "_index": "fake-news",
        "_id": "WYHnvIwBSKJFig-fpYGO",
        "_score": 9.875764,
        "_source": {
          "date": "February 23, 2016",
          "text": "Pope Francis is now the anti-Christ in the eyes of Donald Trump supporters...",
          "title": " New Hampshire Republican Calls Pope Francis The Anti-Christ (IMAGE)",
          "is_fake": 1,
          "subject": "News"
        }
      }
    }
}

\end{lstlisting}
\end{algorithm}
\newpage


% LIST OF FIGURES
%\listoffigures

% LIST OF TABLES
%\listoftables

%\cleardoublepage

\end{document}